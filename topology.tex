
\documentclass{article}
\usepackage{amsmath, amssymb}

\title{Short Introduction to Topology\\{\small For Computer
    Science Grad Students\\\tiny and other people who don't give a crap
    about topology}\\{\small {\bf DRAFT} version 0.6}}
\author{Mark Jason Dominus ({\tt mjd@plover.com})}

\def\R{{\mathbb R}}
\def\Rn{{\R^n}}
\def\cat#1{{\bf #1}}
\def\ball#1#2{B_{#1}({#2})}
\newcounter{exercisecounter}\setcounter{exercisecounter}{1}
\def\Exercise.#1\par{{\item\small \makebox{\bf Exercise \number\theexercisecounter.\hspace{0.5em}}\unskip\ignorespaces#1\addtocounter{exercisecounter}{1}}}

% \def\comp#1{{#1}^C}
% \def\inter#1{{#1}^\circ}
% \def\clos#1{{#1}^\bullet}
% \def\closclos#1{{#1}^{\bullet\bullet}}
% \def\closcomp#1{{#1}^{C\bullet}}
% \def\compclos#1{{#1}^{\bullet C}}
% \def\intercomp#1{{#1}^{C\circ}}
% \def\compinter#1{{#1}^{\circ C}}

\def\comp#1{\smash{#1}^C\vphantom{#1}}
\def\inter#1{\smash{#1}^\circ\vphantom{#1}}
\def\clos#1{\smash{#1}^\bullet\vphantom{#1}}
\def\closclos#1{\clos{\clos{#1}}}
\def\closcomp#1{\clos{\comp{#1}}}
\def\compclos#1{\comp{\clos{#1}}}
\def\intercomp#1{\inter{\comp{#1}}}
\def\compinter#1{\comp{\inter{#1}}}

\def\imp{\rightarrow}           % Logical implication
\def\bind{{\tt >>=}}            % Haskell Monad bind operator


\begin{document}
\maketitle

\abstract{\tiny This document is a {\bf draft}, circulated only to
attract comment and correction.  It contains numerous errors of
fact, vague overgeneralizations, and misleading implications.  Many
claims about general topological spaces in fact apply only to
Hausdorff spaces.  Much of the Applications section should be
considered placeholders that bear only a vague resemblance to the
correct, accurate explanations.  Readers not already familiar with
topology may come away with severe misapprehensions.  Do not rely on
it for anything.

Please send comments, suggestions, and corrections to the author at
{\tt mjd@plover.com}.

Please do not distribute this draft after {\bf 15 December, 2010}.  The
most recent version is available at {\tt http://blog.plover.com/math/topology-doc.html}.

}

\bigskip
%\begin{center}
%{\small (These notes contain at least one error.  How many can you find?)}
%\end{center}
\parskip 1ex

Topology is the branch of mathematics that tries to understand
continuity and continuous functions.  You probably don't need to know
much topology for your CS degree, but people will bring it up from
time to time.  For example, you will hear in your category theory
class that topological spaces form a category with continuous
functions as the arrows.  Or you will hear that certain topological
spaces are natural models for intuitionistic logic.  The goal of these
notes is to be the minimal explanation of topology that will enable
you to understand those two things.

On the real line, continuity is defined in terms of distance: a
function from $\R$ to $\R$ is {\em continuous\/} if points that were
sufficiently close together in the domain are mapped to points that
are close together in the codomain.  The use of ``close together''
implies a distance function, which in $\R$ is simply $|x-y|$.  But what
about spaces that are more complex, for which distance doesn't make
sense, or where no distance function is apparent?

Topology reformulates this idea of ``closeness'' in terms of special
sets called ``open sets''.

\section*{Open sets}

The single fundamental concept of topology is the ``open set''.

You probably already know that in $\R$, an {\em open interval} is a set
of the form $\{x : a < x < b \}$ for some real $a$ and $b$.  This
interval is written $(a, b)$.  An {\em open set in $\R$} is a union of
open intervals.  The union might be infinite.  For example, $\{ x : x
> 0 \}$ is an open set because it is the union of the intervals $(0,
n)$ for all positive $n$.  $\R$ itself is open, and the empty set is
also considered open, because it's an empty union of intervals.

This generalizes to $\Rn$ as follows: An {\em open ball\/} is the set
of all points $x$ whose distance from some center point $p$ is less
than some radius $\epsilon$.  In $\R$, an open ball is nothing but an
open interval.  (In $\R^3$, open balls are actually ball-shaped.)  An
open set in $\Rn$ is a union of open balls.

% The important thing about an open set is that points can be very close
% to the set without being in it.  For example, consider $I^\circ = (0,
% 1)$.  0 is not in $I$, but you can find points that are in $I$
% arbitrarily close to 0.  This is quite different from the {\em
%   closed\/} interval $I = \{ x : 0 \le x \le 1 \}$.  There is no point
% outside of $I$ with the analogous property.  So open sets give us a
% way to talk about ``closeness'' without having to concern ourselves
% directly with distance measures.

Open sets are topology's generalization of the analytic notion of
``close together''.  In topology, two entities are ``close'' if every
open set containing one intersects the other one.  They are boundedly
far apart if there are disjoint open sets around them.

\section*{Topologies}

A {\em topological space\/} is a set $X$ equipped with an open set
structure.  That is, we pick out some of the subsets of $X$ and
designate them as ``open''.  The collection of ``open'' subsets of $X$
that we designate is called the {\em topology\/} of the space.
Since a topology is supposed to be a collection of open sets, its
members are required to satisfy certain properties that are analogous
to those satisfied by open sets in $\Rn$:

\label{topology-def}
\begin{enumerate}
\item $X$ itself must be open, and $\varnothing$ must be open
\item \label{open-u} Any union of open sets must be open
\item \label{open-i} The intersection of any two open sets must be open
\end{enumerate}

Property \ref{open-i} implies that any finite intersection
of open sets must be open, by induction.  But infinite intersections
might {\em not\/} be open.  This matches the behavior of $\Rn$, where finite
intersections of open sets are open, but infinite intersections are
sometimes not.  For example, the intersection of the open sets $(-x,
x)$ for all positive reals $x$ is the set $\{0\}$, which is not
open.  In contrast, the union described in property \ref{open-u} can be any
union whatever, even a huge uncountable union.

This is obviously a very general definition, and the usual practice
here would be to present a number of examples, some of which would be
very weird.  I am not going to do that.  $\Rn$ is the only example you
need to remember.  But keep in mind that the definition {\em is\/}
very general, and applies in all sorts of surprising places and
spaces.  The end of these notes will discuss a couple of examples of that.

With the definition in place we can define a number of topological
properties, such as interiors, boundaries, limits, continuous
functions, and so on, which are generalizations of those properties in
$\Rn$.

I will present a few of these, and then talk about computer science
and logic.

\section*{Connected sets}

A ``connected'' set is one that is in all one piece.  For example,
intervals in $\R$ are connected, and $\R$ itself is connected, but the
set $(0,1)\cup(2,3)$ is not (it's in two pieces) and the set of
integers is not (it's in many pieces).

To prove that a set is not connected, we find a ``separation'' of it:
we split the set into two nonempty, disjoint pieces that have some
space in between.  Mere disjointness is not sufficient for separation.
After all, we can split $\R$ itself into two pieces $\{x : x<0\}$ and
$\{x : x\ge0\}$ which are disjoint. But the two pieces touch each
other, and their union is $\R$, which is connected.  We want to say
that in a separation the two pieces are boundedly far apart.

The way we express this in topology is to say that the two pieces are
separated if they are contained in disjoint open sets.  We can't find
disjoint open sets containing $\{x : x<0\}$ and $\{x : x\ge0\}$,
because every open set that contains $\{x : x\ge0\}$ must also contain
some negative numbers.  And one can show that $\R$ is connected, as it
should be.  In fact it follows directly from the definition of $\R$
that it is connected.\footnote{If you know the definition of $\R$ in
terms of Dedekind cuts, you should be able to see this.}

% Don't we require a stronger property than I said here, namely that
% each of the two open sets excludes the closure of the other?  In
% normal spaces my formulation is equivalent.  Should I mention this?
% Maybe in a footnote?

So let's define a connected set: A {\em separation} of a set $S$ is a
partition of $S$ into two nonempty disjoint parts, $X$ and $Y$, such that there
are disjoint open sets $X'$ and $Y'$ with $X\subset X'$ and $Y \subset
Y'$.  A set is {\em disconnected\/} if there is a separation of it,
and {\em connected\/} if not.

% Whenever you see a definition in topology, the first thing you should
% do is to think about what it means in $\R$.  Topology was invented to
% help mathematicians understand the structure of $\R$.  $\R$ is the first
% and most important example of every elementary topological concept.

Under this definition, is $\R\backslash\{0\}$ connected or
disconnected?  Disconnected, of course.  The separation is obvious:
$X$ is the positive reals, $Y$ is the negative reals, $X'=X$ is an
open set containing $X$, and $Y'=Y$ is an open set containing $Y$ that
is disjoint from $X'$.

\begin{itemize}
\Exercise. A finite subset of $\R$ is connected only if it has fewer than two elements.

\Exercise. $\mathbb Q$, the set of rationals, is not connected.

\end{itemize}

%% There are other ways to define connectedness.  For example, we might
%% say that a set $S$ is connected if, for every $p, q\in S$, there is a
%% continuous mapping $f : [0, 1] \mapsto S$ with $f(0) = p$ and $f(1) =
%% q$.  Topologists call this property ``path-connectedness''.  It is
%% less general than the definition I gave before because it fails for
%% certain extremely large connected sets.  Imagine a space that looks
%% locally like $\R$, but instead of having a countable number of copies of
%% $[0, 1)$ stuck together, it has an uncountable number of copies.

\section*{Limits}

Let $S$ be a set in a topological space.  A point $p$ is called a {\em
  limit point\/} of $S$ if it is arbitrarily close to the rest of $S$.
As usual, the notion of ``arbitrarily close to'' is formulated in
terms of open sets.  The formal definition is that a point $p$
(whether in $S$ or not) is a limit point of $S$ if every open set that
contains $p$ also intersects some point of $S$ other than $p$.

Let's think about limit points in $\R$.  We have some set, say
$I^\circ = (0,1)$.  When is $p$ a limit point of $I^\circ$?  Well, every point
$x$ in $I^\circ$ is a limit point, because every open set containing
$x$ contains some open interval around $x$, which then intersects
$I^\circ$ in some point close to but not equal to $x$.

But also, 0 and 1 are limit points of $I^\circ$, because every open
interval around 0 intersects $I^\circ$, and 1 similarly.

No other points of $\R$ are limit points of $I^\circ$.  Consider $-1$
for example.  Not every open set containing $-1$ intersects $I^\circ$.
$(-1.5, -0.5)$ is a counterexample.

% Limit points are sometimes called ``accumulation points'', because
% they're the places near which many of the points of the set $S$ pile up.

Let's consider another example, $S = \{1, \frac12, \frac13, \ldots\}$. 
None of the points of $S$ are limit points, because for each
point $p\in S$, it's possible to find a small open interval around $p$
that does not intersect $S$ anywhere else.  But the point 0 is a limit
point of $S$, because every open interval around 0 intersects some
points of $S$.  0 is indeed the limit of the sequence $\left< 1, \frac12,
\frac13, \ldots\right>$, which explains the name ``limit point''.

\begin{itemize}
\Exercise. A finite subset of $\R^n$ has no limit points.

\Exercise. If $S$ is a sequence in $\Rn$ that converges to $p$, then $p$
is a limit point of the range of $S$.

\Exercise. But the range of $S$ might have limit points even if $S$ does not converge.

\Exercise. Which points of $\R^2$ are limit points of an open disc?
Which points are limit points of a closed disc?

\end{itemize}

As we saw with $I^\circ$, limit points of a set might be in the set,
or not.  A set which contains all its limit points is called a {\em
  closed\/} set.  The union of a set $S$ with its set of limit points
is called the {\em closure\/} of $S$, written $\clos S$.  The name
``closed'' was chosen because, in $\R$, the closure of an open interval
$(a, b)$ is the the closed interval $[a, b]$, and in $\Rn$, the closure of
an open ball is a closed ball.

\begin{itemize}
\Exercise. Some sets are both open and closed.

\Exercise. Some sets are neither open nor closed.

\Exercise. The closure of a set is a closed set.

\Exercise. If $S$ is a closed set, then $\clos S = S$.

\Exercise. Every closed set $S$ is the closure of some set $S'$. 

\end{itemize}

If $A\subset B$, and $\clos A = B$, we say that
$A$ is {\em dense in $B$}: every point of $B$ is ``close to'' many
points of  $A$.  The canonical example of this is:

\begin{itemize}
\Exercise. $\mathbb Q$ is dense in $\R$.

\end{itemize}

Probably the most important theorem about closed sets is that a set is
closed if and only if its complement is open.    This is
true on $\R$ for the usual meanings of ``closed'' and ``open''.  For
example, the complement of the closed interval $[0, 1]$ is the open
set $(-\infty, 0) \cup (1, \infty)$.

This theorem is so important that it's the only one I'm going to prove
in this whole document.  Let $G$ be an open set and $H$ be its
complement.  I want to show that $H$ is closed.  Let $p$ be a limit
point of $H$.  I want to show that $p\in H$.  Since $p$ is a limit
point of $H$, every open set containing $p$ must intersect $H$.  $G$
is an open set not intersecting $H$, so $p\not\in G$, so $p\in H$.

Now conversely, suppose $H$ is a closed set and $G$ is its
complement.  I want to show that $G$ is open.  For each $p\in G$, $p$
is not a limit point of $H$, so there is some open set $G_p$ that
contains $p$ and does not intersect $H$.  Since each $G_p$ is disjoint
from $H$, each is a subset of $G$.   Let $G'=\bigcup G_p$.  
$G'\subset G$, because $G'$ is a union of subsets of $G$.  
$G\subset G'$ because $p\in G_p\subset G'$ for each $p\in G$.
So $G=G'$.  $G'$ is a
union of open sets, and thus is open, and therefore so is $G$.

\begin{itemize}
\Exercise. The closure of a set $S$ is the smallest closed set $C$
with $S\subset C$.

\Exercise. The closure of a set $S$ is the intersection of all closed sets $C$
with $S\subset C$.


\end{itemize}

One reason that closed sets are important in analysis because of the
{\em Heine-Borel theorem:\/} A continuous function on a closed,
bounded set has bounded values.  This is not true of continuous
functions on bounded sets in general.  For example, the function $x
\mapsto 1/x$ on the bounded set $(0, 1)$ is continuous but not
bounded.

The topological version of the Heine-Borel theorem replaces ``closed,
bounded set'' with ``compact set''.  (In $\Rn$, these are equivalent.)
Compactness is probably the most important topological concept that
I'll omit from these notes.

\subsection*{Boundaries}

We saw that sometimes some of the limit points of a set are in the
set, and some are not.  But the limit points that are not in the set
are close to it.  That is the source of the topological definition of
a boundary: The {\em boundary\/} of a set $S$ is the set of points
that are limit points of both $S$ and its complement.

\begin{itemize}
\Exercise. The boundary of an open interval $(a, b)$ is the set $\{a, b\}$.

\Exercise. The boundary of a closed interval $[a, b]$ is the set $\{a, b\}$.

\Exercise. The boundary of the open ball $\{(x,y,z) : x^2 + y^2 + z^2 <
1\}$ in $\R^3$ is the sphere  $\{(x,y,z) : x^2 + y^2 + z^2 = 1\}$.   So
is the boundary of the closed ball. 

\Exercise. The boundary of an open set $S$ is $\clos S - S$.

\end{itemize}

Topology also defines the {\em interior\/} of a set, which is dual to
the closure of the set.  The interior of a set $S$, written $\inter
S$, is the points in $S$ that are {\em not\/} on the boundary.

\begin{itemize}
\Exercise. $\inter S$ is open.

\Exercise. If $S$ is open, then $S = \inter S$.

\Exercise. $\inter S$ is the largest open set $G$ with $G\subset S$.

\Exercise. The interior of a closed ball in $\Rn$ is an open ball.  

\Exercise. Write $\comp S$ for the complement of set $S$.  $\compinter S = \closcomp S$, and
$\compclos S = \intercomp S$.

\end{itemize}

\section*{Continuity}

In analysis, a function is continuous if you can make the image in the
codomain as small as you like, by choosing a small enough part of the
domain.  That is, a function $f$ is continuous if, given a subset $C$
of the codomain that is contained in some small open set $C'$, we can
find a small open set $D'$ in the domain such that $p\in D'$ implies
that $f(p)\in C$.  This motivates the following topological definition
of continuity:

Suppose $A$ and $B$ are topological spaces.  A function $f : A\to B$
is {\em continuous\/} if $f^{-1}(R)$ is open whenever $R$ is open.
($f^{-1}(R)$ is the set of all points of the domain of $f$ that map to
points of $R$.  That is, $f^{-1}(R) = \{ x : f(x)\in R \}$.)

\begin{itemize}
\Exercise. The identity function is continuous.

\Exercise. Constant functions are continuous.

\Exercise. Compositions of continuous functions are continuous.

\Exercise. Continuous functions of connected sets have connected images.

\Exercise. Continuous functions of open sets are {\em not\/} necessarily open.

\Exercise. The analytic definition of continuity on $\Rn$ coincides with the
topological definition.

\end{itemize}

\subsection*{Homeomorphic spaces}

If there's a bijective mapping between two sets $A$ and $B$ that is
continuous in both directions (that is, both $f$ and $f^{-1}$ are
continuous) then there is a bijection between the open sets of $A$ and
$B$, and so they have the same open set structure.  Since all
topological properties are formulated in terms of open sets, $A$ and
$B$ have exactly the same topological properties, and are
topologically equivalent.  The topology jargon for this is that they
are {\em homeomorphic\/}, and the bijective mapping is a {\em 
homeomorphism\/}. 

Topologists do not distinguish between different homeomorphic
spaces.

\begin{itemize}
\Exercise.  Even though $(0,1)$ and $\R$ are geometrically quite
different, they are topologically equivalent.

\end{itemize}

\section*{Applications}

\subsection*{The category of topological spaces}

One basic example of a category is the category $\cat{Top}$ of
topological spaces.  The objects are topological spaces, that is, each
object is a set $X$ equipped with a topology $T$, which is a
collection of open sets satisfying the open-sets properties I listed
back on page \pageref{topology-def}.  The arrows are continuous
functions between these spaces.

When we have an arrow $f : X\to Y$ and an arrow $g : Y\to X$ that are
inverses, then $f$ and $g$ are bijections, and the spaces are
homeomorphic.  In the categorial view, the objects are isomorphic.
Indeed, this is one of the main examples of isomorphic objects.  They
may be different sets, but a topologist doesn't care.  

% For example, let $X$ be $\R^2$ with the usual topology.  Let $Y$ be
% the set of points $Y = \{(x, y, z)  \in \R^3 : x^2 + y^2 + z^2 = 1 \}
% \backslash \{ (0,0,1) \}$, with the rule that a subset $G$ of this is open
% whenever there is an open set $G'\subset\R^3$ with $G=G'\cap Y$.  That
% is, it's a sphere with the north pole removed, and the open sets are
% just the ordinary open sets of $\R^3$, restricted to the sphere.
% These spaces are homeomorphic.  They are both connected.  Neither is
% compact.  Etc.  Any topological property possessed by one space is
% possessed by the other, and any topological property lacked by $X$ is
% also lacked by $Y$.

This category was an important motivator for category theory in the
first place.  Suppose you want to show that two topological spaces,
say a torus and a sphere, are not homeomorphic.  One way is to find a
topological property that one has that the other does not.  

An important topological property is the ``fundamental group'' of a
space.  Without getting into too many details, one chooses a ``base
point'' $x$ in the space, and considers the set of loops that start
and end at $x$. Loops are considered equivalent if one can be smoothly
transformed into the other.  It turns out that the choice of base
point doesn't matter, at least for connected sets.  One can then put a
group structure on the equivalence classes of loops in a
straightforward way.  The group operation is essentially concatenation
of loops.  This group is called the fundamental group of the space.

On a sphere, all the loops fall into a single equivalence class,
because any loop can be contracted smoothly down to a single point.
So the resulting group has only one element.\footnote{A space whose
  fundamental group is trivial is called ``simply connected''.}

But the fundamental group of a torus is quite different.  Consider a
very small loop $a$ on the side of the torus.  Such a loop can be
shrunk to a point.  But now consider a larger loop $b$ that starts on
the outside edge, goes around through the hole, and comes back to the
outside. Such a loop cannot get any smaller, so $a$ and $b$ are
different.  This shows that the group is nontrivial.  In fact the
fundamental group of the torus is $\mathbb Z^2$.  This proves that
the torus is not a sphere.

% Would disc vs circle be a better example here?  It is much easier to
% see that the fg of the circle is Z than that the fg of the torus is z^2.

I bring this up because this fundamental group construction is
precisely a functor, one which maps the category of topological spaces
to the category of groups. It is one of the very first functors ever
considered as such.  The study of such functors from $\cat{Top}$ to
various categories of algebraic structures such as $\cat{Grp}$ is
called {\em algebraic topology\/}.

\subsection*{Models of intuitionistic logic}

Classical logic can be modeled with boolean algebras.  The typical
approach is to take the trivial boolean algebra, which has only two
elements, $\top$ (true) and $\bot$ (false).  A {\em valuation\/} is an
assignment of one of these values to each propositional variable.
This then extends to a truth value for every propositional formula: 

\begin{itemize}
\item $v(f\vee g) = \bot$, if $v(f) = \bot$ and $v(g) = \bot$;
  otherwise $\top$
\item $v(f\wedge g) = \top$, if $v(f) = \top$ and $v(g) = \top$;
  otherwise $\bot$
\item $v(f\imp g) = \bot$, if $v(f) = \top$ and $v(g) = \bot$;
  otherwise $\top$
\item $v(\neg f) = \bot$, if $v(f) = \top$; otherwise $\bot$
\end{itemize}

A formula is a {\em tautology\/} if it has a value of $\top$
regardless of how you assign values to the variables.  That is,
tautologies are those formulas to which every valuation assigns a
value of $\top$.  The tautologies are precisely the theorems of
classical logic.

This is nothing more than a formalization of the method of truth
tables.  Each line in the truth table represents one valuation.  You
calculate the value of your formula for each valuation, and if it is
$\top$ on every line, the formula is a tautology.

One can generalize this.  Choose a set $X$.  We will assign valuations
which are subsets of $X$.  Extend the valuation to formulas by saying
that $v(f\vee g) = v(f)\cup v(g)$ and $v(f\wedge g) = v(f)\cap v(g)$.
Say that tautologies are those which are assigned the value $X$ by all
such valuations.

But nobody bothers to do this, because the tautologies are the same
regardless of which nonempty set $X$ you use.  So there is no reason
to bother with any $X$ bigger than one element, with $\top = X$ and
$\bot = \{\}$.    

Subsets of $X$ form a boolean algebra, in which the boolean operations
are set union and set intersection.  But classical logic is faithfully
modeled by the simple boolean algebra with only two elements.

In intuitionistic logic, the situation is somewhat different.  Many
formulas are theorems of classical logic but not of intuitionistic
logic.  The most well-known example is probably $\neg p\vee p$.
This gets the value $\top$ for all boolean algebra valuations, but it
is not a theorem of intuitionistic logic.  So boolean algebras are not
faithful models of intuitionistic logic.  Other important examples of
formulas that are classical tautologies but that are not theorems of
intuitionistic logic are $\neg\neg p\imp p$ and $((p\imp q)\imp p)\imp
p$.

Intuitionistic logic cannot be modeled by boolean algebras.  Instead,
one can model intuitionistic logic with a generalization of boolean
algebras called {\em Heyting algebras\/}.  But no finite Heyting
algebra suffices: to model formulas with up to $n$ variables requires
a Heyting algebra of size at least $2^{2^n}$.

But there is a simple example of an infinite Heyting algebra that does
successfully model intuitionistic logic.  One takes the values of
propositional variables to be subsets of $\R$, then extends this to a
valuation on all formulas as follows:

\begin{itemize}
\item$ v(f\vee g) = v(f)\cup v(g)$
\item$ v(f\wedge g) = v(f)\cap v(g)$
\item$ v(f\imp g) = \inter{(\comp{v(f)} \cup v(g))}$
\item$ v(\neg f) = \intercomp{v(f)}$
\end{itemize}

The notation $\comp{F}$ here means the complement of the set
$F$, and $\inter{F}$ means the topological interior of the
set $F$.
The value given to
$f\imp g$ is the {\em topological interior\/} of the set of points
that are  either in $v(g)$ or not in $v(f)$.  

With this definition of valuation, theorems of intuitionistic logic
are precisely the formulas which are assigned a value of $\R$ by every
valuation. 

For example, one can show that every valuation of $\neg(p\wedge\neg
p)$ is $\R$.  Suppose $v(p)$ is some set $P\subset\R$.  Then
$v(\neg(p\wedge\neg p)) = \intercomp{(P\cap\intercomp P)}$.  But
$\intercomp P \subset \comp P$, so $P\cap\intercomp P \subset
P\cap\comp P = \varnothing$.  Then $v(\neg(p\wedge\neg p)) =
\intercomp\varnothing = \inter\R = \R$, regardless of what set $P$ was
assigned as the value of of $p$.  Thus $\neg(p\wedge\neg p)$ is a
tautology, and it is indeed a theorem of intuitionistic logic.

On the other hand, $\neg p\vee p$, which is not a theorem, is not a tautology.
Say that $v(p)$ is $(0, \infty)$, the positive reals.  Then $v(\neg p\vee p)
= v(\neg p) \cup v(p)
= \intercomp{(0, \infty)} \cup (0, \infty)
= \inter{(-\infty, 0]}    \cup (0, \infty) 
= (-\infty, 0)            \cup (0, \infty)
\ne\R$.

This is the intuitionistic analogue of the truth-table method. For
more complete details about this, see S\o rensen and Urzyczyn,
{\em Lectures on the Curry-Howard Isomorphism\/}, sections 2.3--4.

\subsection*{The compactness principle}

An important theorem of model theory is the so-called ``compactness
principle''.  Suppose you have an infinite set of axioms,
which are sentences of first-order logic.  And suppose there is a
model for any finite subset of these axioms.  The compactness
principle says there must also be a model for the entire set of
axioms, and so the axioms must be consistent.

Since I didn't go into detail about compactness, I can't discuss this
in detail.  But the models can be taken to be boolean algebras, which
are themselves examples of compact topological spaces.  (The topology
is that {\em every\/} subset of the space is open; such spaces are
called ``discrete''.)

If one has models for some subsets of the axioms, then one can exhibit
a model for the union of these subsets: it is the product of the
topological spaces that are models for the subsets.  But this only
works if one can interpret this product space as a boolean algebra.
This requires that the product of compact topological spaces be
compact, which is called the Tychonoff theorem.

Here is an important application of the compactness principle, which
also illustrates why it might be surprising.  Consider the usual
axioms of the real numbers, such as $\forall a,b : a+b = b+a$;
$\forall x,y : \exists z : x\cdot z > y$, and so on.  Add to these
axioms the following infinite set of axioms concerning the constant
$\epsilon$: $\epsilon < 1$, $\epsilon < \frac12$, $\epsilon <
\frac13$, and so on. Still no problem.  Now add $\epsilon > 0$.

Clearly there is a model for any finite subset of these.
Because any finite subset of these axioms is required only to satisfy
some of the usual properties of $\R$, plus the requirement that
$\epsilon < \frac1n$ for some integer $n$, plus possibly that
$\epsilon$ is positive.  So take the model to be $\R$, and $\epsilon =
\frac1{n+1}$.

But by the compactness principle, there must be a model for all of the
axioms at once, and here $\R$ will not do, because there is no such
$\epsilon$ in $\R$.  

So the compactness principle tells us that there must be a model that
satisfies all the usual properties of $\R$, but which also has an
infinitesimal element $\epsilon$ which is still positive, but smaller
than $\frac1n$ for every integer $n$.  And since there is a model for
this set of axioms, they must be consistent.  The compactness principle
proves immediately that the existence of an infinitesimal number
$\epsilon$ is not inconsistent with the other properties of the reals.

What we get in this case is called the {\em hyperreals\/} or {\em
  nonstandard reals\/}, the subject of the branch of mathematics
called {\em nonstandard analysis\/}.  

\subsection*{Monads are closure operators}

The topological closure operator satisfies the following properties:

\begin{enumerate}
  \item $S\subset\clos S$
  \item $\closclos S\subset \clos S$
  \item $(S\subset T) \rightarrow (\clos S\subset\clos T)$
\end{enumerate}

Analogously, monads in Haskell are required to support the following
functions:

\begin{enumerate}
  \item {\tt return :: s -> M s }
  \item {\tt join :: M M s -> M s }
  \item {\tt fmap :: (s -> t) -> (M s -> M t)}
\end{enumerate}

(They are also required to support a `bind' operation called
\bind, but {\tt a~\bind~b} = {\tt join (fmap b a)} and {\tt join~x} =
{\tt  x \bind\ id}, so the two formulations
are equivalent.)  

Categorially, these are instances of the same structure, called a {\em
  Kleisli triple\/} or often just a {\em monad\/}.  More formally, a
monad is a functor (in the category-theory sense) equipped with
two natural transformations.  In the programming language category,
the endofunctor is a type constructor; in the topological space
category it is a closure operator.  The natural transformations
manifest themselves in the programming language category as the {\tt
  join} and {\tt return} functions, and in the topological space
category as the required closure properties.

Categorial monads were first studied because of their connections with
topological closure operators.  Their significance in programming
languages only became clear later on.  Closure operators were earlier
shown by Kuratowski to be fundamental to topological spaces: there is
an equivalent formulation of topological spaces in terms of closure
operators instead of in terms of open sets.

There's also a close connection between topological closure operators
and modal operators in modal logic, but I don't know yet what it is.

\section*{Further reading}

A standard and commonly recommended text is {\em Topology\/}, by James
Munkres.  I think it's one of those books that is confusing and
obscure even when you already know what it's going to say, but a lot
of people do seem to like it.

My own favorite is {\em General Topology\/}, by John~L. Kelley.  It's
extremely terse, but always clear and pithy.  The appendix contains a
brilliantly clear exposition of axiomatic set theory, if you're
interested in that.

As usual, the {\em Schaum's Outline\/} series volume is clear, direct,
and unpretentious.  It's declass\'e, but so are these notes.

\end{document}

